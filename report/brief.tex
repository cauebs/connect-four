%-------------------------------------------------------------------------------
\documentclass{article}

%-------------------------------------------------------------------------------
% Packages
\usepackage{amsmath}
\usepackage[portuguese]{babel}
\usepackage{othelloboard}

%-------------------------------------------------------------------------------
% User-commands
\newcommand{\todo}[1]{{\color{red}{#1}}}

%-------------------------------------------------------------------------------
% Project configs
\title{Relatório de I.A.: Gomoku, Parte 1}
\author{Cauê Baasch de Souza \\
        João Paulo Taylor Ienczak Zanette}
\date{\today}

%-------------------------------------------------------------------------------
\begin{document}
    \maketitle{}

    \section{Modelo de estados e do jogo}

    O projeto foi modelado como:

    \begin{description}
        \item [Estados:] Cada estado é uma Matriz (implementada como um
            \textit{array} bi-dimensional), em que cada elemento indica a
            presença de uma pedra e, quando há a presença efetiva, de qual
            jogador.

        \item [O jogo:] Uma estrutura que contém o estado atual do tabuleiro,
            os dois jogadores (dados como instâncias de qualquer tipo que
            obedeça uma interface padrão \texttt{Player}), um indicador
            marcando de quem é a vez, e o número total de jogadas até o
            momento.
    \end{description}

    \section{Heurística e Função de Utilidade}

    Para formular a heurística, são adotadas as seguintes categorias de
    cenários:

    \begin{description}
        \item [Vitória iminente:] Quando, se na vez da I.A., a vitória é
            garantida apenas colocando uma única peça no tabuleiro. As
            possibilidades desse cenário são quando há:
            \begin{itemize}
                \item Uma quádrupla (não necessariamente contígua) com espaço
                    para virar uma quíntupla (Figura~\ref{win-four}).
                \item Duas triplas (não necessariamente contíguas) com um
                    espaço compartilhado (Figura~\ref{win-shared-triple}).
                \item Dois espaços separados apenas por uma permutação de 4
                    elementos $\langle X, X, X, E \rangle$, em que X é uma peça
                    do jogador atual (Figura~\ref{win-four-permut}).
            \end{itemize}

        \item [Possível vitória:] Qualquer caso semelhante aos de vitória
            iminente, porém com uma peça do jogador substituída por um espaço
            vazio.
    \end{description}

    Sendo assim, a função de heurística adotada está descrita na
    Equação~\ref{heuristic-function}:

    \begin{equation}
        H(s) = \begin{cases}
            \infty, & \mbox{Se há vitória iminente} \\
            C_p * Q_p + \sum_{e \in E} C_e * N_p, & \mbox{Em qualquer outro caso}
        \end{cases}
        \label{heuristic-function}
    \end{equation}

    Em que:

    \begin{itemize}
        \item $s$ é um estado do jogo.
        \item $C_x$ indica uma constante relacionada à categoria de $x$.
        \item $N_p$ indica o número de peças de \todo{noideawhat}.
        \item $Q_p$ \todo{eu realmente não lembro}.
        \item $E$ \todo{não lembro o que descreve, mas é o conjunto de $e$'s}.
    \end{itemize}

    \section{Estruturas adicionais}

    \todo{TERMINAR ESTA CAVALA IMUNDA.}

    \section{Otimizações planejadas}

    \todo{TERMINAR ESTA CAVALA IMUNDA.}

    \begin{figure}[h]
        \centering
        \caption{Exemplo de quádrupla contígua. No exemplo, o
        jogador garante sua vitória quando posicionar uma peça
        em $(1, a)$ ou $(1, f)$.\label{win-four}}
        \begin{othelloboard}{1}
            \dotmarkings{}
            \drawboardfromstring{-XX-XX--}
        \end{othelloboard}
    \end{figure}

    \begin{figure}
        \centering
        \caption{Exemplo de permutação de $\langle X, X, X, E
        \rangle$. No exemplo, posicionando uma peça em $(1, c)$
        garante vitória no turno seguinte, sem chances de o
        oponente escapar.\label{win-four-permut}}
        \begin{othelloboard}{1}
            \dotmarkings{}
            \drawboardfromstring{%
                -X-XX---%
            }
        \end{othelloboard}
    \end{figure}

    \begin{figure}
        \centering
        \caption{Exemplo de duas triplas não-contíguas. No
        exemplo, posicionando uma peça em $(1, a)$ garante
        vitória no turno seguinte, sem chances de o oponente
        escapar.\label{win-shared-triple}}
        \begin{othelloboard}{1}
            \dotmarkings{}
            \drawboardfromstring{%
                -X-XX---%
                --------%
                X-------%
                X-------%
                X-------%
            }
        \end{othelloboard}
    \end{figure}
\end{document}
